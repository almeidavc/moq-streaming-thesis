\chapter{\abstractname}

\ac{HAS} is the de facto standard for delivering media content at scale over the Internet. Nevertheless, \ac{HAS} systems, which were originally developed for \ac{VOD}, are poorly suited for low-latency live streaming. In particular, these systems are slow at adapting and responding to congestion. Due to the pull-based approach to streaming, clients must explicitly downgrade the stream quality, which increases the delay until the lower quality video segments are sent, when the network throughput can't keep up with the stream's bitrate. Furthermore, with TCP, any video segments which have been already requested have to be fully downloaded. Additionally, \ac{HAS} systems use a bitrate ladder with a limited number of bitrate steps. The network bandwidth may not match a bitrate from the bitrate ladder exactly, such that the system can't fully utilize the available network resources.

To address the limitations of \ac{HAS}, Media over QUIC aims to develop a scalable protocol designed for media distribution that leverages QUIC to achieve lower latencies. In this thesis, we implement a prototype live-streaming system using \ac{MoQ}, and examine streaming protocols that use different prioritization strategies to achieve low latencies. One such strategy is to divide the stream into a base layer and an optional enhancement layer that can be deprioritized when the network is congested. We explore a concrete implementation of this approach that consists in transmitting B-frames using lower priority QUIC streams. We also explore a different strategy which consists in prioritizing new over old media. To achieve this, we use a separate stream for each \ac{GoP}, and prioritize newer \acp{GoP} over older ones. We implement a testbed and compare our streaming protocols.

Our results show that deprioritizing B-frames achieves better performance than transmitting frames in the order they are encoded when the network bandwidth drops slighly below the stream's bitrate. However, deprioritizating B-frames does not decrease latency by a meaningful amount for network profiles that are more representative of real-world network conditions, since B-frames don't contribute much to the total bitrate of a video stream. On the other hand, prioritizing new \acp{GoP} over old ones reduces latency significantly, especially when the network bandwidth is below the stream's bitrate, resulting in a maximum latency close to the \ac{GoP} duration.