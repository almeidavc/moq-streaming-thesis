\chapter{\abstractname}

HTTP Adaptive Streaming (HAS) is the de facto standard for delivering media content at scale over the Internet. Nevertheless, HAS systems, which were originally developed for Video on Demand (VOD), are poorly suited for low-latency live streaming because they are slow at adapting and responding to congestion.

To address the limitations of HAS, Media over QUIC (MoQ) aims to develop a scalable protocol designed for media distribution that leverages QUIC to achieve lower latencies. In this thesis, we implement a prototype live streaming system using MoQ and examine various prioritization strategies for reducing latency. One strategy is to divide the stream into a base layer and an optional enhancement layer, which can be deprioritized when the network is congested. We explore a concrete implementation of this approach that consists in transmitting B-frames using lower priority QUIC streams. Another strategy we explore is to prioritize newer video segments over older ones. We implement a testbed to simulate various network profiles and measure the reductions in latency achieved by these strategies. 

Our results show that deprioritizing B-frames achieves better performance than transmitting frames in their encoded order when the network bandwidth drops slightly below the stream's bitrate. However, deprioritizing B-frames does not lead to a significant latency reduction for network profiles that more accurately represent real-world network conditions since B-frames don't contribute much to the bitrate of the video stream. On the other hand, prioritizing newer video segments over older ones reduces latency significantly, especially when the network bandwidth is below the stream's bitrate, resulting in a maximum latency close to the segment duration.