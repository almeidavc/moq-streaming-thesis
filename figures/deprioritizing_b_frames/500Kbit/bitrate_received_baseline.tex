\pgfplotsset{
    every legend to name picture/.style={
        legend style={
        draw=none,
        fill=none,
        legend columns=-1,
        % /tikz/every even column/.append style={column sep=0.2cm},
        % font=\tiny,
        at={(0.5,1.02)},
    },
    legend cell align={left},
    },
}

\begin{tikzpicture}
    \begin{axis}[
    width=0.45\textwidth,
    area style,
    axis lines=left,
    ylabel=Bitrate received (Kbit/s),
    title=Baseline,
    legend columns=-1,
    legend to name=named,
    scaled ticks=false,
    ymin=0, ymax=800, ytick distance=200, ytickmin=200,
    xtick distance=0.5, minor x tick num=1,
    x filter/.code={\pgfmathparse{#1/60}}, % Convert seconds to minutes
    xticklabel={ % Split into hours and minutes
        \pgfmathsetmacro\hours{floor(\tick)}%
        \pgfmathsetmacro\minutes{(\tick-\hours)*0.6}%
        % Use some trickery to get leading zeros
        \pgfmathprintnumber{\hours}:\pgfmathprintnumber[fixed, fixed zerofill, skip 0.=true, dec sep={}]{\minutes}%
    },
    % legend style={
    %     draw=none,
    %     fill=none,
    %     legend columns=-1,
    %     % /tikz/every even column/.append style={column sep=0.2cm},
    %     % font=\tiny,
    %     at={(0.5,1.02)},
    % },
    % legend cell align={left},
    ]
        \addplot+[fill=blue!20, draw=blue!40] table[meta=bitrateReceivedKbit] {data/sample_runs_gop15/500Kbit/baseline/total_bitrate.dat} \closedcycle;
        \addplot+[fill=green!20, draw=green!40] table[meta=bitrateReceivedKbit] {data/sample_runs_gop15/500Kbit/baseline/I_frames_bitrate.dat} \closedcycle;
        \addplot+[fill=orange!20, draw=orange!40] table[meta=bitrateReceivedKbit] {data/sample_runs_gop15/500Kbit/baseline/P_frames_bitrate.dat} \closedcycle;
        \addplot+[fill=purple!20, draw=purple!40] table[meta=bitrateReceivedKbit] {data/sample_runs_gop15/500Kbit/baseline/B_frames_bitrate.dat} \closedcycle;
        \addplot[ultra thick, red, no marks] coordinates {
        (0, 9999)
        (0, 500)
        (120, 500)
        (120, 9999)
        };
        \legend{Total, I-frames, P-frames, B-frames}
    \end{axis}
\end{tikzpicture}